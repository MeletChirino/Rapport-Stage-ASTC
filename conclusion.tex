\section{Conclusion}
ASTC m'a permis d'intégrer son équipe de développement de software embarqu\'e et cela a \'et\'e une expérience enrichissant dans laquelle j'ai mis en pratique toutes mes connaissances acquis pendant mon parcours académique.

Les vitesses de développement de software et hardware sont différents et tr\`es souvent le hardware ralentisse le software. VLAB donne la possibilit\'e d'aller plus loin dans le développement de software en faisant des banc de test virtuelles pour valider le fonctionnement des SWC même si le hardware n'existe pas.

Non seulement j'ai appris des solutions technologies pour l'industrie mais encore j'ai appris des nouvelles méthodes de travaille, nouvelles techniques de programmation, versionnage avec des outils comme git et des serveurs entreprise, modélisation des composants pour simulations avanc\'ees, entre autres connaissances qui seront utiles pour moi pour tout ma vie professionnelle.

Pour conclure, ce stage m’a permis d’évoluer dans une structure multinationale à taille humaine, et de m’intégrer dans l’équipe de développement des logiciels embarqués, de leur apporter mes connaissances et de profiter de leurs expériences. J’ai aussi appris à m’adapter aux exigences et les valeurs de l’entreprise quelles sont liées au professionnalisme, autonomie, le partage d’information avec le tissage des relations humaines.


%La soci\'et\'e ASTC Design Partners est une soci\'et\'e qui travaille dans le domaine des systemes embarqu\'es en développant des outils pour ses partenaires industrials comme BMW, Ford, Bosch, Infineon, entre autres. Dans cet stage j'ai apris le fonctionement interne des microntroleurs et appareils utilis\'es dans le domaine embarqu\'e. J'ai aussi apris les protocole de travailler dans une enterprise qui a des fortes relations avec des autres soci\'et\'es a l'étranger et l'importance de parler l'anglais dans le monde moderne.

%Les activit\'es d'integration et validation d'une plateforme virtuelle avec un logiciel AUTOSAR m'a appris des nouvelles méthodes de programmation pour mieux écrire mon code et comment le documenter en utilisant un outil versionnage comme git.