\section{Conclusion}
ASTC m’a permis d’intégrer son équipe de développement de software embarqué et cela a été une expérience enrichissant dans laquelle j’ai mis en pratique toutes mes connaissances acquis pendant mon parcours académique.

Les vitesses de développement de software et hardware sont différents et très souvent le hardware ralentisse le software. VLAB donne la possibilité d’aller plus loin dans le développement de software en faisant des bancs de test virtuelles pour valider le fonctionnement des SWC même si le hardware n’existe pas. 

Non seulement j’ai appris des solutions technologies pour l’industrie mais encore j’ai appris des nouvelles méthodes de travail, nouvelles techniques de programmation, versionnage avec des outils comme git et des serveurs entreprise, modélisation des composants pour simulations avancées, entre autres connaissances qui seront utiles pour moi pour tout ma vie professionnelle. 


Pour conclure, ce stage m’a permis d’évoluer dans une structure multinationale à taille humaine, et de m’intégrer dans l’équipe de développement des logiciels embarqués, de leur apporter mes connaissances et de profiter de leurs expériences. J’ai aussi appris à m’adapter aux exigences et les valeurs de l’entreprise quelles sont liées au professionnalisme, autonomie, le partage d’information avec le tissage des relations humaines. 