% This LaTeX was auto-generated from an M-file by MATLAB.
% To make changes, update the M-file and republish this document.

\documentclass[12pt]{article}
\usepackage{natbib}
\usepackage{minted}
\usepackage[french]{babel}
\usepackage[utf8x]{inputenc}
\usepackage[smartEllipses]{markdown}
\usepackage{lipsum}
\usepackage{amsmath}
\usepackage{hyperref}
\usepackage{tabularx}
\usepackage{multirow}
\usepackage{graphicx}
\usepackage[final]{pdfpages}
\usepackage[T1]{fontenc}
\usepackage{pgffor}
\usepackage{listings}
%\usepackage[colorinlistoftodos]{todonotes}
\usepackage{xcolor}
\usepackage[nottoc,numbib]{tocbibind}% Para que la bibliografia salga en el table of contents
\colorlet{Mycolor1}{green!10!orange!90!}
\usepackage{subfigure}
\sloppy
\definecolor{lightgray}{gray}{0.5}
\setlength{\parindent}{0pt}
\setlength{\parskip}{1em}

\bibliographystyle{abbrv}

\begin{document}
\title{Rapport stage ASTC}
\author{CHIRINO CAICEDO Melet}
\begin{titlepage}

\newcommand{\HRule}{\rule{\linewidth}{0.5mm}} % Defines a new command for the horizontal lines, change thickness here

\center % Center everything on the page
 
%----------------------------------------------------------------------------------------
%	HEADING SECTIONS
%----------------------------------------------------------------------------------------

\textsc{\LARGE Université Toulouse III}\\[0.5cm] % Name of your university/college
\textsc{\Large Paul Sabatier}\\[1.0cm] % Major heading such as course name
\textsc{\large Rapport du stage 2021/2022}\\[0.5cm] % Minor heading such as course title

%----------------------------------------------------------------------------------------
%	TITLE SECTION
%----------------------------------------------------------------------------------------

\HRule \\[0.4cm]
{ \huge \bfseries Integration d'un logiciel embarqu\'e AUTOSAR en simulation}\\[0.4cm] % Title of your document
\HRule \\[1.5cm]
 
%----------------------------------------------------------------------------------------
%	AUTHOR SECTION
%----------------------------------------------------------------------------------------

\begin{minipage}{0.4\textwidth}
\begin{flushleft} \large
\markboth{Pojet en \LaTeX}
\emph{Auteur:}\\
\textsc{CHIRINO CAICEDO}\\ % Your name
Melet David
%\textsc{RIBEIRO} Bruno % Your name
\end{flushleft}
\end{minipage}
~
\begin{minipage}{0.5\textwidth}
\begin{flushright} \large
\emph{Encadrant industriel:} \\
\textsc{BROUEILH} Nicolas \\ % Supervisor's Name
\emph{Encadrant universitaire:} \\
\textsc{RIVI\`ERE} Nicolas  % Supervisor's Name
\end{flushright}
\end{minipage}\\[2cm]

% If you don't want a supervisor, uncomment the two lines below and remove the section above
%\Large \emph{Author:}\\
%John \textsc{Smith}\\[3cm] % Your name

%----------------------------------------------------------------------------------------
%	DATE SECTION
%----------------------------------------------------------------------------------------

{\large \today}\\[1cm] % Date, change the \today to a set date if you want to be precise

%----------------------------------------------------------------------------------------
%	LOGO SECTION
%----------------------------------------------------------------------------------------

\includegraphics[width=6in]{img/Logo_UT3.jpg}\\[2cm] % Include a department/university logo - this will require the graphicx package
 
%----------------------------------------------------------------------------------------

\vfill % Fill the rest of the page with whitespace

\end{titlepage}

%\{Contents}
\tableofcontents

\newpage

%-------------------------- * Introduccion * --------------------------------
\newpage
Aqui empiezo hablando de que es astc, que es una empresa internacional que hace herramientas de simulacion de microcontroladores. Explico un poco que es vlab, un par de capturas de pantalla con la que se muestre sus caracteristicas funcionales y zas, juera. 

Finalmente digo que mi stage se trata de modelizar un gateway ya existente con la herramienta de simulacion.

\begin{figure}[!htb]
 \centering
 \includegraphics[width=\textwidth]{img/gateway_block_diagram.png}
 \caption{Gateway Block diagram}
 \label{fig:block diagram}
\end{figure}

%-------------------------- * Descripcion * --------------------------------
\newpage
\subsection{Description du projet}
La societe Infineon Technologie GA propone un gateway securis\'e KIT\_A2G\_TC377\_SEC\_GTW \cite{gateway} (fig \ref{fig:gw-photo}) el cual corre con un microcontrolador \textit{AURIX TC37xEXT} para poder aumentar la conectividad de diferentes redes como el FlexRay, LIN, CAN y Ethernet dentro del automobil. El sistema operativo usado dentro de este gateway es fourni por Vector Informatik y se llama MICROSAR Classic \cite{vector.microsar}. Este sistema operativo es totalmente compatible con AUTOSAR lo cual nos deja con una ECU 100\% funcional sobre la cual se pueden correr software components (swc) compatibles con AUTOSAR. Ademas tambien se utilisa un switch ethernet securis\'e Marvell 88Q5050 que permite velocidad y ancho de banda estables con toda la seguridad que un automobil requiere.

\begin{figure}[!htb]
 \centering
 \includegraphics[width=0.8\textwidth]{img/secure-gateway.jpg}
 \caption{KIT\_A2G\_TC377\_SEC\_GTW}
 \label{fig:gw-photo}
\end{figure}

El gateway viene con un programa de demostracion que pretende hacer la prise en main del gateway, testear el funcionamiento basico y mostrar las capacidades del mismo. Para testear el Demo es necesario conectar 2 ECU's externar a ciertos puertos especificados en la figura \ref{fig:connections-diagram}. El demo en cuestion tiene 2 use Cases los cuales vamos a explicar por separado.

\begin{itemize}
    \item Use Case 1 : En este Use Case se envia un frame can y el gateway le hace forward por puerto ethernet. Retratado en la figura \ref{fig:gw-demo-uc1}.
    \item Use Case 2 : En Este Use case se envia un frame ethernet y esto se le hace forward hacia un bus can. Retratado en la figura \ref{fig:gw-demo-uc2}.
\end{itemize}

\begin{figure}[!htb]
 \centering
 \includegraphics[width=0.5\textwidth]{img/GWUseCase1.png}
 \caption{Gateway Demo Use Case 1}
 \label{fig:gw-demo-uc1}
\end{figure}

\begin{figure}[!htb]
 \centering
 \includegraphics[width=0.5\textwidth]{img/GWUseCase2.png}
 \caption{Gateway Demo Use Case 2}
 \label{fig:gw-demo-uc2}
\end{figure}

\subsection{Objectifs du projet}

La propuesta de ASTC Desing Partners es virtualizar este Gateway para poder agilizar el proceso del desarrollo de software, sobretodo en un contexto de escasez mundial de microcontroladores. Ademas, es bien sabido que suele haber pocas unidades disponibles para testear por lo que es mejor tenerlo digitalizado para que cada programador pueda avanzar por su lado sin necesidad de hacer cola por el hardware, a menos que sea necesario.

Bien si vlab puede correr el microcontrolador, este gateway tiene otros componentes con los que se interactua y por tanto puede suponer un problema si no estan presentes asi que otro de los objetivos es modelizar circuitos y buses de datos que esten conectados entre si para que el software pueda arrancar.
\begin{itemize}
    \item Modelar componentes necesarios para hacer arrancar el gateway
    \item Hacer un testbench con los componentes necesarios conectados
    \item Configurar Vlab para que corra MICROSAR Classic
\end{itemize}
%
%-------------------------- * Objetivos * --------------------------------
% Esta parte esta explicada en la problematique
%\section{Objectifs}
%Mis objetivos... Debo preguntarle a Nico por mis objetivos precisos pero yo diria 
\begin{itemize}
    \item cargar microsar sobre un tc37x simulado y hacer que arranque
    \item Imitar las funcionalidades de canoe
    \item Hacer una UI (en pygame papu, eso me va a quedar para toda la vida)
\end{itemize}

%-------------------------- * Estado del Arte * --------------------------------
%\section{Etat de l'Art}
%\subsection{Les outils}
\subsubsection{VLAB}
Vlab permite correr un microcontrolador virtualizado con su respectiva pieza de software dentro y todo lo que eso implica. Tambien permite ver el Codigo fuente del microcontrolador y ejecutarlo paso a paso. Tambien se pueden ver los registros y poner traces en diferentes piezas fisicas del microcontrolador como los puertos, registros, memoria, etc.
\subsubsection{Vector MICROSAR}

\subsubsection{Trace32}
Trace32 es un programa que permite debuguear Sw en microcontroladores. Este programa permite ejecutar un programa paso a paso viendo todas las funciones ejecutadas y los valores de ciertas variables.

\subsection{Etat du projet}
AUTOSAR: AUtomotive Open System ARchitecture est un standard de programation pour les ECU's des automobiles.


Aqui vas a explicar los conceptos fundamentales para entender todo lo que se va a explicar en el desarrollo como: AUTOSAR, VECTOR Microsar, Virtualizacion, nuestro gateway, redes ETH, CAN, LIN, Fr, vlab, SystemC


%-------------------------- * Conception * --------------------------------
\newpage
\subsection{ECU's Externas}

Para poder probar cualquier cosa del gateway primero se necesitaban un par de componentes externos que interactuaran con este mismo, como se muestra en la figura \ref{fig:devices-diagram} del datasheet del gateway. Por lo tanto, se deben hacer este par de ECU's genericas para que interactuen con este gateway.

\begin{figure}[!htb]
 \centering
 \includegraphics[width=\textwidth]{img/GWDemoConnections.PNG}
 \caption{Gateway Demo Devices}
 \label{fig:devices-diagram}
\end{figure}

Para modelar estas ECU's se utilizo pyton con su api de systemC la cual es permite tener un modelo de bajo nivel que puede interactuar con la ECU del gateway en bajo nivel y no al nivel por encima de la simulacion. A estas ECU's genericas se les doto de todos los nodos de comunicaciones soportados por nuestra companhia. Finalmente las ECU's solo enviaban unos cuantos frames CAN y ethernet y notificaban cuando lo recibian. El siguiente paso fue conectarlas a traves de un testbench con nuestra ECU y todo funciono bien.

\subsection{First Demo}
En el primer demo para hacer la prise en main de AUTOSAR habia un ejemplo el cual debia funcionar con el programa CANoe. Se intento replicar el comportamiento del CANoe haciendole ingenieria inversa a los codigos. Se encontro el CanId necesario para que el ECU recibiera el frame CAN pero al parecer esto sigue un protocolo mas preciso por parte del programa del demo (CANoe). Decidimos dejarlo aqui porque eso no era productivo para el proyecto pero se aprendio muchisimo de autosar y de la forma en como esta estructurado su stack de comunicacion.

%\begin{verbatim}
%/* Appl/GenData/CanIf_Lcfg.c L279 */
%CONST(CanIf_RxPduConfigType, CANIF_CONST) CanIf_RxPduConfig[8] = {  /* PRQA S 1514, 1533 */  /* MD_CSL_ObjectOnlyAccessedOnce */
%    /*Index    RxPduCanId  RxPduMask  UpperPduId                                  RxIndicationFctListIdx   RxPduDlc   MsgType                      
%  { /* 0 */    0x0400u  ,  0x643Fu  , CanNmConf_CanNmRxPdu_CAN_5f8bc0cc_Rx      , 1u               ,       8u         , CANIF_MSG_TYPE_CAN       },
%  { /* 1 */    0x0614u  ,  0x07FFu  , CanTpConf_CanTpRxNPdu_CanTpRxNPdu_e872022a, 2u               ,       8u         , CANIF_MSG_TYPE_NO_FD_CAN },
%  { /* 2 */    0x0610u  ,  0x07FFu  , CanTpConf_CanTpRxNPdu_CanTpRxNPdu_29945216, 2u               ,       8u         , CANIF_MSG_TYPE_NO_FD_CAN },
%  { /* 3 */    0x0501u  ,  0x07FFu  , PduRConf_PduRSrcPdu_PduRSrcPdu_7a86d966   , 3u               ,      64u         , CANIF_MSG_TYPE_CAN       },
%  { /* 4 */    0x0310u  ,  0x07FFu  , PduRConf_PduRSrcPdu_PduRSrcPdu_37e6280b   , 3u               ,       4u         , CANIF_MSG_TYPE_CAN       },
%  { /* 5 */    0x0210u  ,  0x07FFu  , PduRConf_PduRSrcPdu_PduRSrcPdu_9e00b2d3   , 3u               ,       1u         , CANIF_MSG_TYPE_CAN       },
%  { /* 6 */    0x0120u  ,  0x07FFu  , PduRConf_PduRSrcPdu_PduRSrcPdu_01c8980f   , 3u               ,       8u         , CANIF_MSG_TYPE_CAN       },
%  { /* 7 */    0x0110u  ,  0x07FFu  , PduRConf_PduRSrcPdu_PduRSrcPdu_b1d1dd8a   , 3u               ,       6u         , CANIF_MSG_TYPE_CAN       } 
%};
%\end{verbatim}

Primero habia un demo en el que habia un sistema de autosar bastante mas simple para el cual tuvimos que hacer varias cosas antes de arrancar, como hackearle el flexray y eso. Luego creamos un modelo en python que enviara datos por CAN. Despues de leer el stack de comunicacion de Autosar me di cuenta que habian varias cosillas que no tenian logica y al parecer el sw wstaba mal planteado. 

\subsection{Gateway}

Un gateway es un nodo de una red que se comunica con una red exterior, usualmente mas grande. Las ventajas de este gateway es que permite conectar varios protocolos de comunicacion como el Fr, CAN, LIN o ETH, para complementar ECU's que no necesariamente esten programadas para utilizarlos. 

Este gateway en si mismo es una ECU programable con el sistema operativo MICROSAR Classic en el cual podemos preprocesar datos antes de comunicarnos con un protocolo exterior. 

Ademas de esto, este gateway cuenta con un switch ethernet securise que soporta un ancho de banda y velocidades muy altas. Esto permite, por ejemplo, a un sistema de inteligencia artifial de conduccion acceder a una camara de 4k a traves de ethernet (por el ancho de banda grande que este tipo de media necesita) procesar datos y hacer una bien su trabajo, pero al tiempo puede pedir datos o enviar a otra ECU por LIN o CAN sin tener el protocolo fisico integrado. Tambien se pueden recibir muchos datos de internet como estado del trafico o datos de navegacion en tiempo real sin saturar protocolos de comunicacion como el LIN o Fr que tienen muchisimo menos ancho de banda.


\subsection{TC37xEXT}
Al comienzo del proyecto empezamos usando el microcontrolador AURIX TC37x \cite{aurix.tc37x} pero luego leyendo el datasheet del Gateway nos dimos cuenta que en realidad se estaba usando el TC37xEXT \cite{aurix.tc37e} el cual es una version con mas modulos disponibles, entre ellos un modulo CAN y un modulo GETH extras los cuales serian utiles en nuestro proyecto. Otras diferencias se encuentran en la tabla \ref{tab:tc37x_delta}.


% Table generated by Excel2LaTeX from sheet 'Sheet1'
\begin{table}[htbp]
  \centering
    \begin{tabular}{|r||l|l|}
	\hline
	\multirow{2}{*}{Module} & \multicolumn{2}{c|}{Microcontroleur}\\
	\cline{2-3}
	& TC37x & TC37xEXT \\
	\hline \hline
	    RAM & TRAM (cached, non-cached) & EMEM (cached, non-cached) \\
	    \hline
	    MCMCAN modules & 2 & 3 \\
	    \hline
	\hline
    \end{tabular}
  \caption{TC37x Vs TC37xEXT}
  \label{tab:tc37x_delta}
\end{table}


Para modelisar los microcontroladores se utiliza python2, tomamos los modelos de cada modulo (memorias, mcmcan, cpus, etc) y los unimos a un bus con su respectiva start address y luego lo compilamos. Al final agregamos el nuevo microcontrolador al toolbox ya creado para poder tener todo bien clean.

El siguiente paso fue testear el microcontrolador. Para programar en estos microcontroladores se usa BIFACES, un compilador especial para estas CPU's. Para testear las memorias se escribieron datos y se leyeron luego. Para el modulo CAN y GETH se testearon sus memorias RAM y se enviaron un par de mensajes hacia una ECU externa  y se verificaron. Todo salio bien y pudimos avanzar con el proyecto.

\subsection{Switch Marvell 88Q5050}

Para que el demo arranque nosotros el sw hace ciertas verificaciones de sus componentes como buses de datos funcionales y verifica el correcto comportamiento de ciertos componentes. En este contexto, el switch marvell 88Q5050 juega una papel muy importante en este proceso. El microcontrolador tiene una secuencia muy precisa para verificar el correcto comportamiento del switch, incluyendo leer y escribir registros por el BUS MDIO y descargar todos los registros del switch para luego usarlos. La secuencia de inicializacion esta detallada en el anexo \ref{anexe:switch-init}. Si se enceuntran errores en esta secuencia de inicializacion el sistema operativo no va a inicar los puertos de comunicacion ethernet y por tanto los demos del gateway no van a funcionar de la mejor manera.


Un dato interesante de este switch es que ademas de funcionar como switch ethernet, permite un modo de funcionamiento Distributed Switch Architecture el cual permite que una o varias cpus externas le envien comandos a la cpu del switch para modificar registros dentro del switch o para procesar datos antes de enviarlos o cualquier otra cosa. Este modo de comunicacion da muchisima flexibilidad porque con nuestra ECU podemos controlar el funcionamiento a bajo nivel del switch evitando que sea un componente pasivo y dandole el control al desarrollador de todo lo que quiera hacer con el switch.


Otro problema que nos encontramos realizando esta parte esque el modelo de comunicaciones ethernet del toolbox aurix no permitia protocolos SNAP y el protocolo mencionado arriba es un protocolo SNAP asi que toco agregar el soporte para protocolos SNAP en el modelo propio de aurix. Se testeo luego el comportamiento inical y todo estaba funcionando bien.

Para modelar el switch se utilizo systemC. Se modelo el Switch con base a la informacion fournie por parte del fabricante y se siguio con el proyecto.

\subsection{Use Case 1}

En la figura \ref{fig:block diagram} podemos ver que modulos se comunican con que dispositivo. Para el Use case 1 probamos enviando un mensaje CAN desde la ECU2 con el Id encontrado en el sw de autosar y este funciona ya que envia el mensaje a traves de todo el stack de comunicaciones de autosar pero lastimosamente cuando llega al socket adapter se muere porque los puertos no han sido activados, mala configuracion del switch.

\begin{figure}[!htb]
 \centering
 \includegraphics[width=\textwidth]{img/gateway_block_diagram.png}
 \caption{Gateway Block diagram}
 \label{fig:block diagram}
\end{figure}

\begin{figure}[!htb]
 \centering
 \includegraphics[width=\textwidth]{img/GWConnectionsDiagram.png}
 \caption{Gateway Connections Diagram}
 \label{fig:connections-diagram}
\end{figure}

%Ya con el microcontrolador correcto podemos pasar al demo precargado en microsar tiene 2 USe cases en los que primero se envia algo por CAN y se le hace un forward por eth y viceversa.
%Segun la documentacion del gateway, este gateway esta conectado de la forma mostrada en la figura \ref{fig:connections-diagram}. 

Luego muestras que envias un dato desde CAN con el id que es y que no funcionaba porque patata

\subsection{Use Case 2}

Ni idea que voy a poner aqui. No estoy seguro de legar tan lejos.


%-------------------------- * resultados * --------------------------------
\newpage
\section{Resultats} 

Le premier résultat à remarquer est l'inclusion du support pour le \textit{TC37xEXT} dans la version 3.8.11 de l'aurix toolbox en python. Une version pour vlab3 (v 0.4.0) a \'et\'e aussi ajout\'e avec une librarie plus performante appele\'e \textit{hipersim} qui marche totalement sur C++. 

Le switch Marvell a été modélisé pour initialiser les ports dans le logiciel AUTOSAR mais ce développement a pris plus de temps que l'attendu parce que les datasheets du fournisseur sont arrivées en retard. En autre, toolbox Ethernet\footnote{Toolbox développé par la division australienne de la société.} ne supportait pas certains types de protocole et les instructions arrivaient incomplets dans le switch. Le protocole Clause 22 utilis\'e dans le bus MDIO c'était un nouveau protocole qui a pris du temps à être développé.  

%En arrivant à la fin du projet l'aurix encore à des problèmes parce qu'il cherche à se communiquer avec des transceivers \`a travers du switch. La solution propos\'e pour ces composants, vu que ce sont des composant qui se communiquent \`a travers du switch, c'était faire une machine \`a état qui répond le nécessaire pour initialiser les ports Ethernet mais c'est encore en développement et c'est un procès qui peut générer des problèmes au futur. 

Le premier cas d'utilisation a suivi le chemin des données de la figure \ref{fig:autosar-com-stack} : \textit{Microcontroler} $->$ \textit{CAN\_Driver} $->$ \textit{CAN\_IF} $->$ \textit{PDU-R} $->$ \textit{SocketAdapter}\cite{sock_adp_man} $->$ \textit{DETError}\cite{det_man}. Dans le \textit{SocketAdapter} le logiciel trouve un problème parce que les connections ne sont pas initialises. Ce problème des connections touche aussi le deuxième cas d'utilisation donc cela n'a pas pu être test\'e. 

\begin{figure}[!htb] 
\centering 
\includegraphics[width=\textwidth]{img/SoAd_Error.png} 
\caption{Socket adaptor error} 
\label{fig:soad-error} 
\end{figure}


%-------------------

%-------------------------- * conclusion * --------------------------------
\newpage
\section{Conclusion}
ASTC m’a permis d’intégrer son équipe de développement de software embarqué et cela a été une expérience enrichissant dans laquelle j’ai mis en pratique toutes mes connaissances acquis pendant mon parcours académique.

Les vitesses de développement de software et hardware sont différents et très souvent le hardware ralentisse le software. VLAB donne la possibilité d’aller plus loin dans le développement de software en faisant des bancs de test virtuelles pour valider le fonctionnement des SWC même si le hardware n’existe pas. 

Non seulement j’ai appris des solutions technologies pour l’industrie mais encore j’ai appris des nouvelles méthodes de travail, nouvelles techniques de programmation, versionnage avec des outils comme git et des serveurs entreprise, modélisation des composants pour simulations avancées, entre autres connaissances qui seront utiles pour moi pour tout ma vie professionnelle. 


Pour conclure, ce stage m’a permis d’évoluer dans une structure multinationale à taille humaine, et de m’intégrer dans l’équipe de développement des logiciels embarqués, de leur apporter mes connaissances et de profiter de leurs expériences. J’ai aussi appris à m’adapter aux exigences et les valeurs de l’entreprise quelles sont liées au professionnalisme, autonomie, le partage d’information avec le tissage des relations humaines. 

%-------------------

%-------------------------- * Bibliografia * --------------------------------
\newpage
\bibliography{Biblio}
%-------------------

%-------------------------- * Anexos * --------------------------------
\newpage
\subsection{Codes du premier demo}

\subsubsection{Hybrid Node}
\label{anexe:first_demo:hybrid_node}
\inputminted[autogobble]{python}{anexes/first_demo/hybrid_node.py}

\subsubsection{Testbench}
\label{anexe:first_demo:testbench}
\inputminted[autogobble]{python}{anexes/first_demo/testbench.py}

\subsubsection{Run file}
\label{anexe:first_demo:run_file}
\inputminted[autogobble]{python}{anexes/first_demo/run_microsar.py}


\subsection{Codes du demo de la gateway}

\subsubsection{Testbench}
\label{anexe:gw_demo:testbench}
\inputminted[autogobble]{python}{anexes/Gw_demo/testbench.py}

\subsubsection{Run file}
\label{anexe:gw_demo:run_file}
\inputminted[autogobble]{python}{anexes/Gw_demo/run_microsar.py}
%-------------------

%\begin{figure}[!htb]
%\centering
%\includegraphics[width=0.2\textwidth]{images/ubication-sipi.png}
%\caption{Location du village Sipí \cite{sipi:ubicacion}}
%\label{img:sipi:ubicacion}
%\end{figure}

%Para citar la bibliografia haces \cite{Photovoltaic}
%Para referenciar las imagenes haces lo sgte \ref{fig:schema}

%\subsection{Initialisation du switch}
%\begin{figure}[!htb]
 \centering
 \includegraphics[height=0.9\textheight]{img/EthSwt_30_88Q5050_VSwitchInit.png}
 \caption{Sequence d'initialisation du Switch}
 \label{fig:switch-init}
\end{figure}
%\label{anexe:switch-init}


% ---- TC37x Delta ----
%\markdownInput{anexes/TC37xEXT_report.md}
%\label{anexe:tc37ex}

%debes hacer esto para meter archivos de otro lado
%\input{sub_files/Calcul_puissance}
%\label{anexe:calcul-puissance}
%-------------------

%Para meter un pdf haces lo sgte 
%\includepdf{images/Velocidad-Maxima-Energia_13}
%\label{pdf:vent_vitesse_max}
%-------------------

% \begin{figure}
%     \centering
%     \includegraphics[height=\textheight]{Anexes/RadiacionSolar13 (1).pdf}
%     \caption{Caption}
%     \label{fig:radiacion}
% \end{figure}


% Plantilla para colocar muchas imagenes con un multiplot
%\begin{figure}[!htb]
%\centering
%\subfigure[]{\includegraphics [width=2.5in]{lab_2_vision_15.png}}
%\subfigure[]{\includegraphics [width=2.5in]{lab_2_vision_16.png}}
%\subfigure[]{\includegraphics [width=2.5in]{lab_2_vision_17.png}}
%\caption{Paleta de colores}
%\end{figure}

% Plantilla para poner una imagen cualquiera
%\begin{figure}[!htb]
%\centering
%
%\caption{Histograma de la imagen}
%\end{figure}
\end{document}
