\subsection{Les outils}
\subsubsection{VLAB}
Vlab permite correr un microcontrolador virtualizado con su respectiva pieza de software dentro y todo lo que eso implica. Tambien permite ver el Codigo fuente del microcontrolador y ejecutarlo paso a paso. Tambien se pueden ver los registros y poner traces en diferentes piezas fisicas del microcontrolador como los puertos, registros, memoria, etc.
\subsubsection{Vector MICROSAR}

\subsubsection{Trace32}
Trace32 es un programa que permite debuguear Sw en microcontroladores. Este programa permite ejecutar un programa paso a paso viendo todas las funciones ejecutadas y los valores de ciertas variables.

\subsection{Etat du projet}
AUTOSAR: AUtomotive Open System ARchitecture est un standard de programation pour les ECU's des automobiles.


Aqui vas a explicar los conceptos fundamentales para entender todo lo que se va a explicar en el desarrollo como: AUTOSAR, VECTOR Microsar, Virtualizacion, nuestro gateway, redes ETH, CAN, LIN, Fr, vlab, SystemC