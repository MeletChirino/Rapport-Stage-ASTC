\section{Annexes}
\subsection{Codes du premier demo}

\subsubsection{Hybrid Node}
\label{anexe:first_demo:hybrid_node}
\subsubsection*{hybrid\_node.py}
\inputminted[autogobble]{python}{anexes/first_demo/hybrid_node.py}

\subsubsection{Testbench}
\label{anexe:first_demo:testbench}
\subsubsection*{testbench.py}
\inputminted[autogobble]{python}{anexes/first_demo/testbench.py}

\subsubsection{Run file}
\label{anexe:first_demo:run_file}
\subsubsection*{run\_file.py}
\inputminted[autogobble]{python}{anexes/first_demo/run_microsar.py}

\subsection{Codes du TC37xEXT}
\label{anexe:tc37ex}

\subsubsection{Fichiers d'assemblage}
\label{anexe:tc37ex:assembly}
\subsubsection*{defines.py}
\label{anexe:tc37ex:defines}
\inputminted[autogobble]{python}{anexes/tc37ex/defines.py}


\subsubsection*{description.py}
\label{anexe:tc37ex:description}
\begin{par}
Ce fichier description est utilise pour les modeles \textit{tc37x} et \textit{tc37xEXT}.
\end{par}
\inputminted[autogobble]{python}{anexes/tc37ex/description.py}

\subsubsection{Fichiers de compilation}
\label{anexe:tc37ex:compilation}
\begin{par}
Pour compiler vlab utilise \textit{Scons}. Dans les fichiers il faut ajouter les lignes de commandes suivants
\end{par}
\subsubsection*{\_\_bootstrap\_\_.py}
\begin{verbatim}
import aurix.tc37xext
import aurix.tc37xext.__main__
import aurix.tc37xext.sim
import aurix.tc37xext.emem
\end{verbatim}

\subsubsection*{\_\_vlabtoolbox\_\_.py}
\begin{verbatim}
modules.append(
    ("aurix.tc37xext.sim",["TC37xEXT", "Load simulator"])
    )
modules.append(
    ("aurix_scripts.tc37xext_self_test",["TC37xEXT","Run self test"])
    )
\end{verbatim}

\subsubsection{Fichiers de test}
\label{anexe:tc37ex:tests}
\begin{par}
Pour tester l'aurix nous devons executer le fichier \textit{tc37xext\_self\_test.py} et il va executer les tests ecrits en c :
\end{par}
\begin{itemize}
    \item test\_emem\_reg\_access.c
    \item test\_mcmcan\_reg\_access.c
    \item test\_geth\_reg\_access.c
\end{itemize}

\subsubsection*{tc37xext\_self\_test.py}
\label{anexe:tc37ex:tests:self}
\inputminted[autogobble]{python}{anexes/tc37ex/tests/tc37xext_self_test.py}

\subsubsection*{test\_emem\_reg\_access.c}
\label{anexe:tc37ex:tests:emem_reg_access}
\inputminted[autogobble]{c}{anexes/tc37ex/tests/test_emem_reg_access_ext.c}

\subsubsection*{test\_mcmcan\_reg\_access.c}
\label{anexe:tc37ex:tests:mcmcan_reg_access}
\inputminted[autogobble]{c}{anexes/tc37ex/tests/test_mcmcan_reg_access_ext.c}

\subsubsection*{test\_geth\_reg\_access.c}
\label{anexe:tc37ex:tests:geth_reg_access}
\inputminted[autogobble]{c}{anexes/tc37ex/tests/test_geth_reg_access_ext.c}


\subsection{Codes du demo de la gateway}

\subsubsection{MDIO Bus}
\label{anexe:gw_demo:mdio_bus}
\subsubsection*{mdio\_bus.hpp}
\inputminted[autogobble]{c++}{anexes/Gw_demo/mdio_bus.hpp}
\subsubsection*{mdio\_bus.cpp}
\inputminted[autogobble]{c++}{anexes/Gw_demo/mdio_bus.cpp}

\subsubsection{Testbench}
\label{anexe:gw_demo:testbench}
\subsubsection*{testbench.py}
\inputminted[autogobble]{python}{anexes/Gw_demo/testbench.py}

\subsubsection{Run file}
\label{anexe:gw_demo:run_file}
\subsubsection*{run\_file.py}
\inputminted[autogobble]{python}{anexes/Gw_demo/run_microsar.py}
