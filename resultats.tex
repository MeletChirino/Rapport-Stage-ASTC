\section{Resultats}

Le premier résultat a remarquer est l'inclusion du support pour le \textit{TC37xEXT} dans la version 3.8.11 de l'aurix toolbox.


Le switch Marvell a \'et\'e modelis\'e pour initialiser les ports dans le logiciel AUTOSAR mais ce developpement a pris plus de temps que l'attendu parce que les datasheets du fournisseur sont arriv\'es en retard. En autre, toolbox Ethernet\footnote{Toolbox developp\'e par la division australienne de la soci\'et\'e.} ne supportait pas certains types de protocole et les instructions arrivaient incomplets dans le switch. Le protocole Clause 22 utilis\'e dans le bus MDIO c'etait un nouveau protocole qui a pris du temps a \^etre develop\'e. 

En arrivant a la fin du projet l'aurix encore a des probl\`emes parce qu'il cherche \`a se communiquer avec des transceivers \`a travers du switch. La solution propos\'e pour ces composants, vu que ce sont des composant qui se communiquent \`a travers du switch, c'était faire une machine \`a \'etat qui répond le nécessaire pour initialiser les ports Ethernet mais c'est encore en développement et c'est un procès qui peut générer des problèmes au futur.

Le premier cas d'utilisation a suivi le chemin des donn\'ees de la figure \ref{fig:autosar-com-stack} : \textit{Microcontroler} $->$ \textit{CAN\_Driver}\cite{can_if_man} $->$ \textit{CAN\_IF}\cite{can_drv_man} $->$ \textit{PDUR}\cite{pdu_r_man} $->$ \textit{SocketAdapter}\cite{sock_adp_man} $->$ \textit{DETError}\cite{det_man}. Dans le \textit{SocketAdapter} le logiciel trouve un probleme parce que les ports ne sont pas initialis\'es \`a cause des transceivers mention\'es avant. Ce probl\`eme des ports touche aussi le deuxième cas d'utilisation donc cela n'a pas pu \^etre test\'e.
